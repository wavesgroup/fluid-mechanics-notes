\documentclass[12pt]{article}

\usepackage[margin=1in,footskip=0.25in]{geometry}
\usepackage{amsmath}
\usepackage{amssymb}
\usepackage{graphicx}
\usepackage{makeidx}

\makeindex

\begin{document}

\title{Fluid Mechanics for Ocean Scientists}
\author{Milan Curcic}
\date{}

\maketitle

\tableofcontents

\newpage
\section{Introduction}

\subsection{What will you learn in this course}

The course aims to provide students with a solid understanding of fluid
mechanics fundamentals relevant to ocean physics researchers.
By the end of the course, you will be proficient in applying fluid mechanical
concepts and mathematical tools to solve ocean physics research problems.
The course schedule covers a wide range of topics, starting with a review of
vector calculus and progressing through fluid kinematics, velocity gradients,
strain rates, rotation, and stress. It also includes the conservation of
volume, mass, and momentum, as well as stream functions, velocity potentials,
and Bernoulli's principle. Advanced topics such as vorticity, boundary layers,
turbulence, steady Navier-Stokes equations, flow instabilities, rotating and
stratified flows, and both surface and internal gravity waves are also covered.
The course concludes with a review and discussion session.

\subsection{Reference textbooks}

These lecture notes are based on the following textbooks:

\begin{enumerate}
  \item \textit{Fluid Mechanics}, 6th Ed., by Kundu, Cohen, and Dowling
  \item \textit{Essentials of Atmospheric and Oceanic Dynamics} by Geoffrey Vallis
\end{enumerate}

While the notes contain the distilled and required information for you to
succeed in this course, please refer to these textbooks for more detailed
explanations and examples.

\section{Review of vector calculus}

In this section we will review the necessary concepts from vector calculus that
we will use in this course.
These include:
scalars, vectors and tensors;
gradient, divergence, and curl;
line, surface, and volume integrals;
and the Gauss and Stokes theorems.

\subsection{Scalars, vectors, and tensors}

In this course we will mainly use three types of quantities to describe fluid
properties: \textbf{scalars}, \textbf{vectors}, and \textbf{tensors}.

\textbf{Scalars}\index{Scalar} are completely described by their magnitude.
Examples of scalars are temperature, pressure, or density.
A value of 290 K, for example, completely describes the temperature of a fluid
at some point in space and time.
We will write them in equations using italics, e.g. $T$, $p$, $\rho$.

\textbf{Vectors}\index{Vector} have both magnitude and direction.
Examples of vector are velocity, acceleration, or force.
In 3-dimensional Cartesian space with coordinates $(x, y, z)$, for example,
vector $\mathbf{u}(x,y,z)$ can be described by its components

\begin{equation}
  \mathbf{u} =
  \begin{bmatrix}
    u_x \\
    u_y \\
    u_z
  \end{bmatrix}
\end{equation}
where $u_x$, $u_y$, and $u_z$ (each a scalar) are the components of $\mathbf{u}$
in the $x$, $y$, and $z$ directions, respectively.
This is the conventional notation, however, we will often write vectors
inline as $\mathbf{u} = (u_x, u_y, u_z)$.
We will write vectors in equations using boldface, e.g. $\mathbf{u}$,
$\mathbf{a}$, or $\mathbf{F}$.

The magnitude\index{Vector!magnitude}, or norm\index{Vector!norm}, of a 
$\mathbf{u}$ is denoted by $||\mathbf{u}||$, and is calculated as

\begin{equation}
  ||\mathbf{u}|| = \sqrt{u_x^2 + u_y^2 + u_z^2}
\end{equation}

\textbf{Tensors}\index{Tensor} have magnitude, direction, and orientation, such as stress or
strain.
They are vectors that act on each respective surface orthogonal to the direction
of the tensor.
In 3-dimensional space, for example, a stress tensor can be described as:

\begin{equation}
  \mathbf{\tau} =
  \begin{bmatrix}
    \tau_{xx} & \tau_{xy} & \tau_{xz} \\
    \tau_{yx} & \tau_{yy} & \tau_{yz} \\
    \tau_{zx} & \tau_{zy} & \tau_{zz}
  \end{bmatrix}
\end{equation}

It may be useful to think of scalars as 0th-order tensors, vectors as 1st-order
tensors, and tensors as 2nd-order tensors.

\subsection{Unit vectors}

Unit vectors\index{Vector!unit} are vectors with magnitude of 1.
A popular notation for unit vectors in Cartesian coordinates is $\mathbf{i}$,
$\mathbf{j}$, and $\mathbf{k}$, which point in the $x$, $y$, and $z$ directions,
respectively.
So, a vector $\mathbf{u}$ can be written as

\begin{equation}
  \mathbf{u} = u_x \mathbf{i} + u_y \mathbf{j} + u_z \mathbf{k}
\end{equation}

Notice that you can get get the unit vector by dividing any vector by its
magnitude, i.e. $\mathbf{u}/||u||$.

\subsection{Vector operations}

Two vectors can be added, subtracted, or multiplied.
Although vector addition and subtraction are straightforward, vector
multiplication is more interesting.
There are two types of vector multiplication: the \textbf{dot product} and the
\textbf{cross product}.

\subsubsection{Dot product}

The dot product\index{Product!dot} of two 3-dimensional Cartesian vectors
$\mathbf{a}$ and $\mathbf{b}$ is an element-wise sum of their components
(and thus, a scalar!):

\begin{equation}
  \mathbf{a} \cdot \mathbf{b} =
  \begin{bmatrix}
    a_1 \\
    a_2 \\
    a_3
  \end{bmatrix}
  \cdot
  \begin{bmatrix}
    b_1 \\
    b_2 \\
    b_3
  \end{bmatrix}
  = a_1 b_1 + a_2 b_2 + a_3 b_3
\end{equation}

More generally, the dot product of two n-dimensional vectors $\mathbf{a}$ and
$\mathbf{b}$ is

\begin{equation}
  \mathbf{a} \cdot \mathbf{b} = \sum_{i=1}^{n} a_i b_i = a_1 b_1 + a_2 b_2 + \ldots + a_n b_n
\end{equation}

The dot product is commutative, meaning that
$\mathbf{a} \cdot \mathbf{b} = \mathbf{b} \cdot \mathbf{a}$.

It can be useful to think of a dot product as collapsing the two vectors into a
single number contains contributions from each of their components.

\subsubsection{Cross product}

The cross product\index{Product!cross} of two vectors $\mathbf{a}$ and
$\mathbf{b}$ is defined as

\begin{equation}
  \mathbf{a} \times \mathbf{b} = (a_y b_z - a_z b_y) \mathbf{i} +
    (a_z b_x - a_x b_z) \mathbf{j} + (a_x b_y - a_y b_x) \mathbf{k}
\end{equation}

Unlike the dot product, the cross product is anticommutative, meaning that
$\mathbf{a} \times \mathbf{b} = -\mathbf{b} \times \mathbf{a}$.

In fluid mechanics a cross product will often come up when we are interested in
the rotation of a vector field, such as vorticity.

\subsection{Total and partial derivatives}

We will denote total\index{Derivative!total} and partial\index{Derivative!partial}
derivative operators (for example, in time $t$)
as $\frac{d}{dt}$ and $\frac{\partial}{\partial t}$.
Scalars, vectors, and tensors alike can be differentiated with respect to any
variable.
A derivative of a vector is simply a vector of derivatives of its components:

\begin{equation}
  \frac{d\mathbf{u}}{dt}
    = (\frac{d u_x}{d t}, \frac{d u_y}{d t}, \frac{d u_z}{d t})
    = \frac{du_x}{dt} \mathbf{i} + \frac{du_y}{dt} \mathbf{j} + \frac{du_z}{dt} \mathbf{k}
\end{equation}

\begin{equation}
  \frac{\partial \mathbf{u}}{\partial t}
    = (\frac{\partial u_x}{\partial t}, \frac{\partial u_y}{\partial t}, \frac{\partial u_z}{\partial t})
    = \frac{\partial u_x}{\partial t} \mathbf{i} + \frac{\partial u_y}{\partial t} \mathbf{j} + \frac{\partial u_z}{\partial t} \mathbf{k}
\end{equation}
and likewise for tensors.

\subsection{Gradient, divergence, and curl}

Now, we introduce another new operator that builds on top of previous
concepts to describe how scalar and vector fields change in space.
This operator is called \textbf{nabla}\index{Nabla} and is denoted by $\nabla$:

\begin{equation}
  \nabla = \frac{\partial}{\partial x} \mathbf{i} +
    \frac{\partial}{\partial y} \mathbf{j} +
    \frac{\partial}{\partial z} \mathbf{k}
\end{equation}

A good way to think about $\nabla$ is as of a \textit{differential operator},
which itself is a 3-dimensional vector that can operate on scalars or vectors.
Specifically:

\begin{itemize}
  \item $\nabla p$ is as vector that is a gradient of a scalar field $p$;
  it quantifies how $p$ changes in space.
  \item $\nabla \cdot \mathbf{u}$ is a scalar that is the divergence of a vector
  field $\mathbf{u}$; it quantifies how $\mathbf{u}$ flows out of a point.
  \item $\nabla \times \mathbf{u}$ is a vector that is the curl of a vector field
  $\mathbf{u}$; it quantifies how $\mathbf{u}$ rotates around a point.
\end{itemize}

$\nabla$ is also often called \textit{del}.

\subsubsection{Gradient}

The gradient\index{Gradient} of a scalar field $T$ is a vector field that points in the
direction of the greatest rate of increase of $T$.
It is denoted by $\nabla T$ and is defined as

\begin{equation}
  \nabla T = \frac{\partial T}{\partial x} \mathbf{i} +
    \frac{\partial T}{\partial y} \mathbf{j} +
    \frac{\partial T}{\partial z} \mathbf{k}
\end{equation}

\subsubsection{Divergence}

The divergence\index{Divergence} of a vector field $\mathbf{u}$ is a scalar field that describes
the rate at which the vector field flows out of a point.
It is denoted by $\nabla \cdot \mathbf{u}$ and is defined as

\begin{equation}
  \nabla \cdot \mathbf{u} = \frac{\partial u_x}{\partial x} +
    \frac{\partial u_y}{\partial y} + \frac{\partial u_z}{\partial z}
\end{equation}

\subsubsection{Curl}

The curl\index{Curl} of a vector field $\mathbf{u}$ is a vector field that describes the
rotation of the vector field.
It is denoted by $\nabla \times \mathbf{u}$ and is defined as

\begin{equation}
  \nabla \times \mathbf{u} = \left( \frac{\partial u_z}{\partial y} -
    \frac{\partial u_y}{\partial z} \right) \mathbf{i} +
    \left( \frac{\partial u_x}{\partial z} -
    \frac{\partial u_z}{\partial x} \right) \mathbf{j} +
    \left( \frac{\partial u_y}{\partial x} -
    \frac{\partial u_x}{\partial y} \right) \mathbf{k}
\end{equation}

\printindex

\end{document}
