\documentclass[12pt]{article}

\usepackage[margin=1in,footskip=0.25in]{geometry}
\usepackage{amsmath}
\usepackage{amssymb}
\usepackage{graphicx}
\usepackage{hyperref}
\usepackage{makeidx}

\hypersetup{
    colorlinks=true,
    linkcolor=blue,       % Color for internal links
    urlcolor=blue         % Color for external URLs
}

% Number equations, figures, and tables within sections
\numberwithin{equation}{section}
\numberwithin{figure}{section}
\numberwithin{table}{section}

\makeindex

\begin{document}

\title{Fluid Mechanics for Ocean Scientists}
\author{Milan Curcic}
\date{}

\maketitle

\tableofcontents

\newpage
\section{Introduction}

\subsection{What will you learn in this course}

The course aims to provide students with a solid understanding of fluid
mechanics fundamentals relevant to ocean physics researchers.
By the end of the course, you will be proficient in applying fluid mechanical
concepts and mathematical tools to solve ocean physics research problems.
The course schedule covers a wide range of topics, starting with a review of
vector calculus and progressing through fluid kinematics, velocity gradients,
strain rates, rotation, and stress. It also includes the conservation of
volume, mass, and momentum, as well as stream functions, velocity potentials,
and Bernoulli's principle. Advanced topics such as vorticity, boundary layers,
turbulence, steady Navier-Stokes equations, flow instabilities, rotating and
stratified flows, and both surface and internal gravity waves are also covered.
The course concludes with a review and discussion session.

\subsection{Reference textbooks}

These lecture notes are based on the following textbooks:

\begin{enumerate}
  \item \textit{Fluid Mechanics}, 6th Ed., by Kundu, Cohen, and Dowling
  \item \textit{Essentials of Atmospheric and Oceanic Dynamics} (EAOD) by Geoffrey Vallis
  \item \textit{Atmospheric and Oceanic Fluid Dynamics} (AOFD) by Geoffrey Vallis.
  This is the longer and more in-depth variant of EAOD.
  Many figures in the notes are adapted from AOFD.
\end{enumerate}

While the notes contain the distilled and required information for you to
succeed in this course, please refer to these textbooks for more detailed
explanations and examples.

\newpage
\section{Review of vector calculus}

In this section we will review the necessary concepts from vector calculus that
we will use in this course.
These include:
scalars, vectors and tensors;
gradient, divergence, and curl;
line, surface, and volume integrals;
and the Gauss and Stokes theorems.

\subsection{Scalars, vectors, and tensors}

In this course we will mainly use three types of quantities to describe fluid
properties: \textit{scalars}, \textit{vectors}, and \textit{tensors}.

\textit{Scalars}\index{Scalar} are completely described by their magnitude.
Examples of scalars are temperature, pressure, or density.
A value of 290 K, for example, completely describes the temperature of a fluid
at some point in space and time.
We will write them in equations using italics, e.g. $T$, $p$, $\rho$.

\textit{Vectors}\index{Vector} have both magnitude and direction.
Examples of vector are velocity, acceleration, or force.
In 3-dimensional Cartesian space with coordinates $(x, y, z)$, for example,
vector $\mathbf{u}(x,y,z)$ can be described by its components

\begin{equation}
  \mathbf{u} =
  \begin{bmatrix}
    u_x \\
    u_y \\
    u_z
  \end{bmatrix}
\end{equation}
where $u_x$, $u_y$, and $u_z$ (each a scalar) are the components of $\mathbf{u}$
in the $x$, $y$, and $z$ directions, respectively.
This is the conventional notation, however, we will often write vectors
inline as $\mathbf{u} = (u_x, u_y, u_z)$.
We will write vectors in equations using boldface, e.g. $\mathbf{u}$,
$\mathbf{a}$, or $\mathbf{F}$.

The magnitude\index{Vector!magnitude}, or norm\index{Vector!norm}, of a 
$\mathbf{u}$ is denoted by $||\mathbf{u}||$, and is calculated as

\begin{equation}
  ||\mathbf{u}|| = \sqrt{u_x^2 + u_y^2 + u_z^2}
\end{equation}

\textit{Tensors}\index{Tensor} have magnitude, direction, and orientation, such as stress or
strain.
They are vectors that act on each respective surface orthogonal to the direction
of the tensor.
In 3-dimensional space, for example, a stress tensor can be described as:

\begin{equation}
  \boldsymbol{\tau} =
  \begin{bmatrix}
    \tau_{xx} & \tau_{xy} & \tau_{xz} \\
    \tau_{yx} & \tau_{yy} & \tau_{yz} \\
    \tau_{zx} & \tau_{zy} & \tau_{zz}
  \end{bmatrix}
\end{equation}
In this notation and index ordering, i.e. $\tau_{ij}$, the first index ($i$)
refers to the direction of the stress component, and the second index ($j$)
refers to the direction of the normal to the surface.
In other words, each row of the tensor contains the three components of a
vector, and each column contains the three surface normals that the stress
component is acting on.
For example, $\tau_{xy}$ is the stress in the x-direction and is acting on the
surface whose normal is in the y-direction (and which lies in the x-z plane).

One special type of tensor is the \textit{identity tensor}\index{Tensor!identity}
$\mathbf{I}$, which is a tensor that maps a vector onto itself.
In Cartesian coordinates, it is given by:

\begin{equation}
  \mathbf{I} =
  \begin{bmatrix}
    1 & 0 & 0 \\
    0 & 1 & 0 \\
    0 & 0 & 1
  \end{bmatrix}
\end{equation}

It may be useful to think of scalars as 0$^{th}$-order tensors, vectors as
1$^{st}$-order tensors, and tensors as 2$^{nd}$-order tensors.\\

\subsection{Unit vectors}

Unit vectors\index{Vector!unit} are vectors with magnitude of 1.
A popular notation for unit vectors in Cartesian coordinates is $\mathbf{i}$,
$\mathbf{j}$, and $\mathbf{k}$, which point in the $x$, $y$, and $z$ directions,
respectively.
So, a vector $\mathbf{u}$ can be written as

\begin{equation}
  \mathbf{u} = u_x \mathbf{i} + u_y \mathbf{j} + u_z \mathbf{k}
\end{equation}

Notice that you can get get the unit vector by dividing any vector by its
magnitude, i.e. $\mathbf{u}/||u||$.

\subsection{Vector operations}

Two vectors can be added, subtracted, or multiplied.
Although vector addition and subtraction are straightforward, vector
multiplication is more interesting.
There are two types of vector multiplication: the \textit{dot product} and the
\textit{cross product}.

\subsubsection{Dot product}

The dot product\index{Product!dot} of two 3-dimensional Cartesian vectors
$\mathbf{a}$ and $\mathbf{b}$ is an element-wise sum of their components
(and thus, a scalar!):

\begin{equation}
  \mathbf{a} \cdot \mathbf{b} =
  \begin{bmatrix}
    a_1 \\
    a_2 \\
    a_3
  \end{bmatrix}
  \cdot
  \begin{bmatrix}
    b_1 \\
    b_2 \\
    b_3
  \end{bmatrix}
  = a_1 b_1 + a_2 b_2 + a_3 b_3
\end{equation}

More generally, the dot product of two n-dimensional vectors $\mathbf{a}$ and
$\mathbf{b}$ is

\begin{equation}
  \mathbf{a} \cdot \mathbf{b} = \sum_{i=1}^{n} a_i b_i = a_1 b_1 + a_2 b_2 + \ldots + a_n b_n
\end{equation}

The dot product is commutative, meaning that
$\mathbf{a} \cdot \mathbf{b} = \mathbf{b} \cdot \mathbf{a}$.

The magnitude of a dot product of two vectors is equal to the product of their
magnitudes and the cosine of the angle $\theta$ between them:

\begin{equation}
  \mathbf{a} \cdot \mathbf{b} = ||\mathbf{a}|| ||\mathbf{b}|| \cos{\theta}
\end{equation}
To visualize this relationship, take one vector and project it onto the other.
This projection is the magnitude of the vector times the cosine of the angle
between them.
Now one vector and the projection of the other onto the first vector are
pointing in the same direction, so their dot product is the product of their
magnitudes.
It can be useful to think of a dot product as collapsing the two vectors into a
single number contains contributions from each of their components.

\subsubsection{Cross product}

The cross product\index{Product!cross} of two vectors $\mathbf{a}$ and
$\mathbf{b}$ is defined as:

\begin{equation}
  \mathbf{a} \times \mathbf{b} =
  det \begin{bmatrix}
    a_x & a_y & a_z \\
    b_x & b_y & b_z \\
    \mathbf{i} & \mathbf{j} & \mathbf{k}
  \end{bmatrix}
\end{equation}
where $det(\mathbf{M})$ means the \textit{determinant of matrix $\mathbf{M}$}.

Using the so-called \textit{rule of Sarrus}, the cross product can be calculated
as:

\begin{equation}
  \mathbf{a} \times \mathbf{b} = (a_y b_z - a_z b_y) \mathbf{i} +
    (a_z b_x - a_x b_z) \mathbf{j} + (a_x b_y - a_y b_x) \mathbf{k}
\end{equation}
or:

\begin{equation}
  \mathbf{a} \times \mathbf{b} =
    \begin{bmatrix}
      a_y b_z - a_z b_y \\
      a_z b_x - a_x b_z \\
      a_x b_y - a_y b_x
    \end{bmatrix}
\end{equation}

The result of a cross product is a vector that is orthogonal to both $\mathbf{a}$
and $\mathbf{b}$.
Its orientation in space is determined by the right-hand rule:
if you point your right thumb in the direction of $\mathbf{a}$ and your index
finger in the direction of $\mathbf{b}$, then your middle finger will point in
the direction of $\mathbf{a} \times \mathbf{b}$.

The magnitude of the cross product is equal to the product of the magnitudes of
the two vectors times the sine of the angle between them:

\begin{equation}
  ||\mathbf{a} \times \mathbf{b}|| = ||\mathbf{a}|| ||\mathbf{b}|| \sin{\theta}
\end{equation}
So, the magnitude of the cross product is largest when the two vectors are
orthogonal.

Unlike the dot product, the cross product is anticommutative, meaning that
$\mathbf{a} \times \mathbf{b} = -\mathbf{b} \times \mathbf{a}$.

In fluid mechanics a cross product will often come up when we are interested in
the rotation of a vector field, such as vorticity.

\subsection{Matrix multiplication}

Occasionally, we will need to multiply a vector by a matrix, or, a matrix by a
matrix.
As a vector is a special case of a matrix in which either the number of rows or
columns is 1, the same rules of matrix multiplication will apply when we
multiply a vector by a matrix or a matrix by a matrix.
These operations are not commutative, meaning that the order of multiplication
matters.

Take two matrices $\mathbf{A}$ and $\mathbf{B}$ such that

\begin{equation}
  \mathbf{A} =
  \begin{bmatrix}
    a_{11} & a_{12} & a_{13} \\
    a_{21} & a_{22} & a_{23} \\
    a_{31} & a_{32} & a_{33}
  \end{bmatrix}
\end{equation}
and:

\begin{equation}
  \mathbf{B} =
  \begin{bmatrix}
    b_{11} & b_{12} & b_{13} \\
    b_{21} & b_{22} & b_{23} \\
    b_{31} & b_{32} & b_{33}
  \end{bmatrix}
\end{equation}

The result of their multiplication is a matrix $\mathbf{C}$ given by:

\begin{equation}
  \mathbf{C} = \mathbf{A} \mathbf{B} =
  \begin{bmatrix}
    a_{11} b_{11} + a_{12} b_{21} + a_{13} b_{31} &
    a_{11} b_{12} + a_{12} b_{22} + a_{13} b_{32} &
    a_{11} b_{13} + a_{12} b_{23} + a_{13} b_{33} \\
    a_{21} b_{11} + a_{22} b_{21} + a_{23} b_{31} &
    a_{21} b_{12} + a_{22} b_{22} + a_{23} b_{32} &
    a_{21} b_{13} + a_{22} b_{23} + a_{23} b_{33} \\
    a_{31} b_{11} + a_{32} b_{21} + a_{33} b_{31} &
    a_{31} b_{12} + a_{32} b_{22} + a_{33} b_{32} &
    a_{31} b_{13} + a_{32} b_{23} + a_{33} b_{33}
  \end{bmatrix}
\end{equation}

That is, the entry $c_{ij}$ of the product is obtained by multiplying
term-by-term the entries of the $i$-th row of $\mathbf{A}$ and the $j$-th column
of $\mathbf{B}$, and summing these products.
In other words, $c_{ij}$ is the dot product of the $i$-th row of $\mathbf{A}$
and the $j$-th column of $\mathbf{B}$.

\subsection{Total and partial derivatives}

We will denote total\index{Derivative!total} and partial\index{Derivative!partial}
derivative operators (for example, in time $t$)
as $\frac{d}{dt}$ and $\frac{\partial}{\partial t}$.
Scalars, vectors, and tensors alike can be differentiated with respect to any
variable.
A derivative of a vector is simply a vector of derivatives of its components:

\begin{equation}
  \frac{d\mathbf{u}}{dt}
    = (\frac{d u_x}{d t}, \frac{d u_y}{d t}, \frac{d u_z}{d t})
    = \frac{du_x}{dt} \mathbf{i} + \frac{du_y}{dt} \mathbf{j} + \frac{du_z}{dt} \mathbf{k}
    = \begin{bmatrix}
        \frac{du_x}{dt} \\
        \frac{du_y}{dt} \\
        \frac{du_z}{dt}
      \end{bmatrix}
\end{equation}

\begin{equation}
  \frac{\partial \mathbf{u}}{\partial t}
    = (\frac{\partial u_x}{\partial t}, \frac{\partial u_y}{\partial t}, \frac{\partial u_z}{\partial t})
    = \frac{\partial u_x}{\partial t} \mathbf{i} + \frac{\partial u_y}{\partial t} \mathbf{j} + \frac{\partial u_z}{\partial t} \mathbf{k}
    = \begin{bmatrix}
        \frac{\partial u_x}{\partial t} \\
        \frac{\partial u_y}{\partial t} \\
        \frac{\partial u_z}{\partial t}
      \end{bmatrix}
\end{equation}
and likewise for tensors.\\

\subsection{Gradient, divergence, and curl}

Now, we introduce another new operator that builds on top of previous
concepts to describe how scalar and vector fields change in space.
This operator is called \textit{del}\index{Del} and is denoted by the symbol
$\nabla$\index{Nabla} (pronounced "nabla"):

\begin{equation}
  \label{eq:nabla}
  \nabla = \frac{\partial}{\partial x} \mathbf{i} +
    \frac{\partial}{\partial y} \mathbf{j} +
    \frac{\partial}{\partial z} \mathbf{k}
\end{equation}

A good way to think about $\nabla$ is as of a \textit{differential operator},
which itself is a 3-dimensional vector that can operate on scalars or vectors.
Specifically:

\begin{itemize}
  \item $\nabla p$ is as vector that is a gradient of a scalar field $p$;
  it quantifies how $p$ changes in space.
  \item $\nabla \cdot \mathbf{u}$ is a scalar that is the divergence of a vector
  field $\mathbf{u}$; it quantifies how $\mathbf{u}$ flows out of a point.
  \item $\nabla \times \mathbf{u}$ is a vector that is the curl of a vector field
  $\mathbf{u}$; it quantifies how $\mathbf{u}$ rotates around a point.
\end{itemize}

Although, strictly speaking, one is a symbol and the other is an operator,
$\nabla$ ("nabla") and "del" are often used interchangeably.

\subsubsection{Gradient}

The gradient\index{Gradient} of a scalar field $T$ is a vector field that points in the
direction of the greatest rate of increase of $T$.
It is denoted by $\nabla T$ and is defined as

\begin{equation}
  \nabla T = \frac{\partial T}{\partial x} \mathbf{i} +
    \frac{\partial T}{\partial y} \mathbf{j} +
    \frac{\partial T}{\partial z} \mathbf{k}
\end{equation}

Gradient of a scalar field is a vector that points in the direction of the
steepest increase of that field, and its magnitude is the rate of that increase.
For example, imagine hiking up a hill; the gradient of the terrain is a vector
that is pointing toward the steepest incline, and its magnitude is the slope
of that incline.

\subsubsection{Divergence}

The divergence\index{Divergence} of a vector field $\mathbf{u}$ is a scalar field that describes
the rate at which the vector field flows out of a point.
It is denoted by $\nabla \cdot \mathbf{u}$ and is defined as

\begin{equation}
  \label{eq:divergence}
  \nabla \cdot \mathbf{u} = \frac{\partial u_x}{\partial x} +
    \frac{\partial u_y}{\partial y} + \frac{\partial u_z}{\partial z}
\end{equation}

Divergence of a vector field is a scalar that describes how much the vector
field is expanding or contracting at a point.
When divergence is negative, it is commonly referred to as convergence.

\subsubsection{Curl}

The curl\index{Curl} of a vector field $\mathbf{u}$ is a vector field that describes the
rotation of the vector field.
It is denoted by $\nabla \times \mathbf{u}$ and is defined as

\begin{equation}
  \nabla \times \mathbf{u} = \left( \frac{\partial u_z}{\partial y} -
    \frac{\partial u_y}{\partial z} \right) \mathbf{i} +
    \left( \frac{\partial u_z}{\partial x} -
    \frac{\partial u_x}{\partial z} \right) \mathbf{j} +
    \left( \frac{\partial u_y}{\partial x} -
    \frac{\partial u_x}{\partial y} \right) \mathbf{k}
\end{equation}

Curl of a vector field is another vector that is orthogonal to the original
vector field and quantifies how much the vector field is rotating around a
point.
When curl is zero, the vector field is said to be \textit{irrotational}.

\subsection{Gauss and Stokes theorems}

The most useful in our work will be variants of the
\textit{Gauss and Stokes theorems}.
The Gauss theorem relates a volume integral of a divergence of a vector field
to a surface integral of that vector field.
The Stokes theorem relates a surface integral of the curl of a vector field to
a line integral of that vector field.
Here, these are merely stated for reference.
We will explore their meaning and application in more detail as we use them
to derive the fundamental equations for fluid flows.

\subsubsection{Gauss theorem}

The Gauss theorem\index{Theorem!Gauss} states that the volume integral of the
divergence of a vector field $\mathbf{u}$ over a volume $V$ is equal to the
surface integral of $\mathbf{u}$ over the surface $A$ that encloses $V$:

\begin{equation}
  \label{eq:divergence_theorem}
  \int_V \nabla \cdot \mathbf{u} dV = \oint_A \mathbf{u} \cdot d\mathbf{A}
\end{equation}

This form of Gauss's theorem is also known as the
\textit{divergence theorem}\index{Theorem!divergence}.
It will come in handy when we derive the conservation of mass (continuity)
equation.

\subsubsection{Stokes theorem}

The Stokes theorem\index{Theorem!Stokes} states that the surface integral of the curl of a
vector field $\mathbf{u}$ over a surface $A$ is equal to the line integral of
$\mathbf{u}$ over the boundary of $A$:

\begin{equation}
  \int_A (\nabla \times \mathbf{u}) \cdot d\mathbf{A} = \oint_{\partial A} \mathbf{u} \cdot d\mathbf{l}
\end{equation}

\subsection{Summary}

In this chapter, we reviewed:

\begin{itemize}
  \item Scalars, vectors, and tensors;
  \item Vector algebra: dot product ($\mathbf{a} \cdot \mathbf{b}$) and cross
  product ($\mathbf{a} \times \mathbf{b}$);
  \item Derivatives: total ($\frac{d}{dt}$) and partial ($\frac{\partial}{\partial t}$);
  \item Gradient, divergence ($\nabla \cdot \mathbf{u}$), and curl ($\nabla \times \mathbf{u}$);
  \item Gauss theorem that relates volume and surface integrals:
  $\int_V \nabla \cdot \mathbf{u} dV = \oint_A \mathbf{u} \cdot d\mathbf{A}$;
  \item Stokes theorem that relates surface and line integrals:
  $\int_A (\nabla \times \mathbf{u}) \cdot d\mathbf{A} = \oint_{\partial A} \mathbf{u} \cdot d\mathbf{l}$.
\end{itemize}

These concepts will serve as the basic building blocks for everything that
follows in the remainder of this course.

\subsection{Exercises}

\begin{enumerate}
  
  \item Pick your favorite programming language (or ask for a recommendation for one).
  Write a program that defines a scalar, a vector, and a tensor, and assign
  numerical values to them.
  Print the values to the screen.
  Is there a difference in how you define them in your program?

  \item Write a program that calculates the dot product of two vectors.
  Please implement your solution using the basic arithmetic operations such as
  addition and multiplication.
  Then, see if your programming language or one of its software libraries
  provides a function to do this.
  Can you verify your implementation by comparing its output to that of the
  library function?

  \item What is the dot product of two orthogonal vectors?
  How about the dot product of a vector with itself?
  Please write out the solution step by step.

  \item Write a program that calculates the cross product of two vectors.
  Please implement your solution using the basic arithmetic operations such as
  addition and multiplication.
  Then, see if your programming language or one of its software libraries
  provides a function to do this.
  Can you verify your implementation by comparing its output to that of the
  library function?

  \item How would you calculate a derivative of a quantity
  (scalar, for example) in a computer program, e.g. $\frac{\partial a}{\partial x}$?
  Consider that you can approximate a derivative as a difference between two
  values of the quantity at two points in space.
  In other words, assume $\partial a \approx \Delta a = a(x_2) - a(x_1)$,
  and similar for $x$.

  \item Write a computer program that calculates the gradient of a scalar field,
  and the divergence and curl of a vector field.
  
  \item Draw example vector fields that are: (a) non-divergent and irrotational,
  (b) divergent and irrotational, (c) non-divergent and rotational, and (d)
  divergent and rotational.
\end{enumerate}

\subsection*{Further reading}

\begin{itemize}
  \item Chapter 2 of \textit{Fluid Mechanics} by Kundu, Cohen, and Dowling
\end{itemize}

\newpage
\section{Fluid kinematics}

Fluid kinematics describe the fluid motion without considering the forces that
cause it.
We will explore two main views of the flow: the \textit{Lagrangian}\index{Lagrangian}
view, which follows individual fluid particles, and the \textit{Eulerian}\index{Eulerian}
view, which observes the flow at fixed points in space.
Although the Eulerian view is more commonly used in the theory and simulation
of fluid flows, the Lagrangian view will be essential in deriving some of the
fundamental equations, as well as for understanding where certain features of
the flow come from.
We will also introduce some useful concepts to describe the flow, namely
the \textit{velocity potential} and the \textit{stream function}.
These two scalar quantities are complementary to the vector field of velocity
and together provide a complete description of the flow.

\subsection{Lagrangian and Eulerian derivatives of a fluid property}

Consider a 3-dimensional quantity $\varphi$ that varies in space and time such that
$\varphi = \varphi(x, y, z, t)$.
Let's find the rate of change of $\varphi$.
Since it depends on $x$, $y$, $z$, and $t$, the rate of change of $\varphi$ along
each of these dimensions must be considered.
So, the total change of $\varphi$ (let's call it $\delta\varphi$) over spatial and
temporal increments $\delta x$, $\delta y$, $\delta z$, and $\delta t$ is the
sum of changes along each of these dimensions:

\begin{equation}
  \delta\varphi = \frac{\partial \varphi}{\partial x} \delta x +
    \frac{\partial \varphi}{\partial y} \delta y +
    \frac{\partial \varphi}{\partial z} \delta z +
    \frac{\partial \varphi}{\partial t} \delta t
\end{equation}
Divide by $\delta t$ to obtain:

\begin{equation}
  \frac{\delta\varphi}{\delta t} = \frac{\partial \varphi}{\partial x} \frac{\delta x}{\delta t} +
    \frac{\partial \varphi}{\partial y} \frac{\delta y}{\delta t} +
    \frac{\partial \varphi}{\partial z} \frac{\delta z}{\delta t} +
    \frac{\partial \varphi}{\partial t}
\end{equation}
Recall the definition of $\nabla$ (Eq. \ref{eq:nabla}), and let the finite
increment $\delta t$ approach $dt$ (and likewise for $\delta x$, $\delta y$, and
$\delta z$), to obtain:

\begin{equation}
  \frac{d\varphi}{dt} = \frac{\partial \varphi}{\partial x} \frac{dx}{dt} +
    \frac{\partial \varphi}{\partial y} \frac{dy}{dt} +
    \frac{\partial \varphi}{\partial z} \frac{dz}{dt} +
    \frac{\partial \varphi}{\partial t}
\end{equation}
Then, recognize that the velocity in each direction is the rate of change of
the position in that direction:

\begin{equation}
  \frac{d\varphi}{dt} =
    \frac{\partial \varphi}{\partial t} +
    u \frac{\partial \varphi}{\partial x} +
    v \frac{\partial \varphi}{\partial y} +
    w \frac{\partial \varphi}{\partial z} w
\end{equation}
Finally, recall the definition of $\nabla$ (Eq. \ref{eq:nabla}) to obtain:

\begin{equation}
  \label{eq:lagrangian_derivative}
  \frac{d\varphi}{dt} = \frac{\partial \varphi}{\partial t} + \mathbf{u} \cdot \nabla \varphi
\end{equation}
The term $\frac{d\varphi}{dt}$ is called the \textit{total derivative}\index{Derivative!total}
of $\varphi$. It is also called a \textit{Lagrangian derivative}\index{Derivative!Lagrangian},
or \textit{material derivative}\index{Derivative!material}, since it follows
the motion of a fluid particle.
The term $\frac{\partial \varphi}{\partial t}$ is called the
\textit{Eulerian derivative}\index{Derivative!Eulerian},
or \textit{partial derivative}\index{Derivative!partial}
of $\varphi$ with respect to time.
The term $\mathbf{u} \cdot \nabla \varphi$ describes how $\varphi$ changes due
to its spatial variation and the flow of the fluid.

Although the term $\mathbf{u} \cdot \nabla \varphi$ is the dot product of
$\mathbf{u}$ and $\nabla \varphi$, the Lagrangian derivative in Eq.
\ref{eq:lagrangian_derivative} can be expressed as an operator:

\begin{equation}
  \frac{d}{dt} = \frac{\partial}{\partial t} + (\mathbf{u} \cdot \nabla)
\end{equation}
The parentheses on the right-hand side indicate that that term acts as an
operator on a field.

\subsection{Lagrangian derivative of a volume}

Consider a fluid parcel with a constant mass but whose volume may change over
time and is $\int_V dV = V$.
The total rate of change of that volume as it moves with the fluid is equal to
the surface integral of the velocity field $\mathbf{u}$ through the surface
$S$ that is bounding the volume $V$:

\begin{equation}
  \frac{d}{dt}\int_V dV = \int_S \mathbf{u} \cdot d\mathbf{S}
\end{equation}
Recall now the divergence theorem (Eq. \ref{eq:divergence_theorem}) to obtain:

\begin{equation}
  \frac{d}{dt}\int_V dV = \int_V \nabla \cdot \mathbf{u} dV
\end{equation}
Now, for a volume parcel so small that $\int_V dV = \Delta V \to 0$, the
velocity divergence can be considered to be constant over the volume, and the
integral can be replaced by the volume itself:

\begin{equation}
  \frac{d\Delta V}{dt} = \Delta V \nabla \cdot \mathbf{u}
  \label{eq:lagrangian_volume_derivative}
\end{equation}

We can derive a similar expression for the rate of change of a fluid property
per unit volume $q$, such that $q \Delta V$ is the amount of that quantity in
a fluid parcel with the volume $\Delta V$.

\begin{equation}
  \frac{d}{dt} (q \Delta V) = \Delta V \frac{dq}{dt} + q \frac{d\Delta V}{dt}
\end{equation}
Recall the material derivative of $\Delta V$ from Eq. \ref{eq:lagrangian_volume_derivative}
to obtain:

\begin{equation}
  \frac{d}{dt} (q \Delta V) = \Delta V \frac{dq}{dt} + q \Delta V \nabla \cdot \mathbf{u}
\end{equation}

\begin{equation}
  \frac{d}{dt} (q \Delta V) = \Delta V \left( \frac{dq}{dt} + q \nabla \cdot \mathbf{u} \right)
  \label{eq:lagrangian_property_derivative}
\end{equation}

This was for a fluid property that is defined per unit volume.
Let's now do the same for some property $\varphi$ that is defined per unit mass,
such that $\varphi \rho \Delta V$ is the amount of that quantity in the fluid
parcel with the volume $\Delta V$ and density $\rho$ (and mass $\rho \Delta V$).

\begin{equation}
  \frac{d}{dt} (\varphi \rho \Delta V) = \rho \Delta V \frac{d\varphi}{dt} + \varphi \frac{d(\rho \Delta V)}{dt}
\end{equation}
However recall that our fluid parcel has constant mass, so $\frac{d(\rho \Delta V)}{dt} = 0$.
Our total derivative becomes:

\begin{equation}
  \frac{d}{dt} (\varphi \rho \Delta V) = \rho \Delta V \frac{d\varphi}{dt}
\end{equation}

The Lagrangian derivative of a volume will come in handy when we derive the
continuity equation in the next chapter.

%\subsection{Velocity potential}

%Velocity potential is defined as a scalar field $\phi$ such that the velocity
%field $\mathbf{u}$ is the gradient of $\phi$:

%\begin{equation}
%  \mathbf{u} = \nabla \phi =
%  \begin{bmatrix}
%    \frac{\partial \phi}{\partial x} \\
%    \frac{\partial \phi}{\partial y} \\
%    \frac{\partial \phi}{\partial z}
%  \end{bmatrix}
%\end{equation}

%\subsection{Stream function}

%Stream function is defined as a scalar field $\psi$ such that the velocity field
%$\mathbf{u}$ is the curl of $\psi$:

%\begin{equation}
%  \mathbf{u} = \nabla \times \psi
%\end{equation}

\subsection{Summary}

In this chapter, we covered:

\begin{itemize}
  \item Lagrangian (material) and Eulerian (field) derivatives;
  the former follows a fluid parcel of constant mass as it moves through
  the flow field, while the latter is the rate of change at a fixed point
  (or volume) in space;
  \item The Lagrangian derivative of volume, as well as of a fluid property per
  unit volume and per unit mass.
\end{itemize}

We'll use these concepts in the next chapter where we derive the equations of
continuity and motion.

\subsection*{Further reading}

\begin{itemize}
  \item Section 1.1 of \textit{EAOD} by Vallis
  \item Chapter 3 of \textit{Fluid Mechanics} by Kundu, Cohen, and Dowling
\end{itemize}

\newpage
\section{Conservation of mass, momentum, and energy}

In this chapter we will derive the fundamental equations for fluid flows:
continuity, momentum, and energy.
We start with the conservation of mass, which is the easiest to derive, but also
arguably the most fundamental.

\subsection{Conservation of mass}
\index{Continuity}

Recall from the previous chapter that we can take at least two perspectives
on the fluid flow: the Lagrangian perspective, which follows a fluid parcel as it
moves through space, and the Eulerian perspective, which observes the flow at
fixed points in space.
We can thus derive the conservation of mass, or commonly known as the
\textit{continuity}, from both perspectives.
Let's start with the Eulerian perspective, as it may seem more intuitive to
derive from first principles

\subsubsection{Eulerian derivation}
\index{Continuity!Eulerian derivation}

Consider a fixed rectangular volume $\Delta V = \Delta x \Delta y \Delta z$ in
three-dimensional space.
The mass of the fluid in this volume is $\rho \Delta V$, where $\rho$ is the
density of the fluid.
The fluid enters the volume through the surfaces of the box, and the rate at
which the mass enters the volume through a surface is given by the product of
the density, the velocity component normal to the surface, and the area of the
surface.
Let's call this velocity $\mathbf{u}$ with components $u$, $v$, and $w$ in the
$x$, $y$, and $z$ directions, respectively.

For simplicity, let's first consider only the $x$-component of the velocity.
This scenario is illustrated in Fig. \ref{fig:continuity1}.
The fluid mass flow rate into the volume through the left face is $\rho u \Delta y \Delta z$,
and the mass flow rate out of the volume through the right face is
$\rho (u + \frac{\partial u}{\partial x} \Delta x) \Delta y \Delta z$.
The net mass change in the control volume must be balanced by the net mass flow
rate into the volume through the left and right faces:

\begin{figure}[h]
  \centering
  \includegraphics[width=0.8\textwidth]{assets/fig_continuity1.pdf}
  \caption{
    Mass conservation in an rectangular Eulerian control volume.
    The mass convergence, $\partial(\rho u)/\partial x$
    (plus contributions in the $y$ and $z$ directions),
    must be balanced by a density decrease.
    This is Fig. 1.1 in AOFD (Vallis, 2017).
  }
  \label{fig:continuity1}
\end{figure}

\begin{equation}
  \int_V \frac{\partial \rho}{\partial t} dV =
  \rho u + \frac{\partial (\rho u)}{\partial x} \Delta x) \Delta y \Delta z - \rho u \Delta y \Delta z =
\end{equation}

\begin{equation}
  \int_V \frac{\partial \rho}{\partial t} dV =
  \frac{\partial (\rho u)}{\partial x} \Delta x \Delta y \Delta z
\end{equation}
Now, we we allow a flow field to have components in the $y$ and $z$ directions,
the equation becomes:

\begin{equation}
  \int_V \frac{\partial \rho}{\partial t} dV =
  \left[\frac{\partial (\rho u)}{\partial x} + \frac{\partial (\rho v)}{\partial y} + \frac{\partial (\rho w)}{\partial z} \right] \Delta V
\end{equation}
Let $\Delta V \to 0$ to such that any field within $\Delta V$ is uniform to obtain:

\begin{equation}
  \frac{\partial \rho}{\partial t} + \nabla \cdot (\rho \mathbf{u}) = 0
  \label{eq:continuity_eulerian}
\end{equation}
This is the continuity equation in the Eulerian reference frame.

We're not constrained to a rectangular, fixed volume, however.
We can derive this equation for an arbitrary control volume using the divergence
theorem.
The total rate of change of that volume as it moves with the fluid is equal to
the surface integral of the velocity field $\mathbf{u}$ through the surface
$S$ that is bounding the volume $V$ (Fig. \ref{fig:continuity2}).
Mathematically, we can express this as:

\begin{figure}[h]
  \centering
  \includegraphics[width=0.7\textwidth]{assets/fig_continuity2.pdf}
  \caption{
    Mass conservation in an arbitrary Eulerian control volume $V$ bounded by a
    surface $S$. The mass increase, $\int_V(\partial \rho/\partial t)dV$
    is equal to the mass flowing into the volume,
    $-\int_S(\rho\mathbf{v}) \cdot d\mathbf{S} = -\int_V \nabla \cdot (\rho\mathbf{v})dV$.
    This is Fig. 1.2 in AOFD (Vallis, 2017).
  }
  \label{fig:continuity2}
\end{figure}

\begin{equation}
  \int_V \frac{\partial \rho}{\partial t} dV = \int_S \rho \mathbf{u} \cdot d\mathbf{S}
\end{equation}
Now, recall the divergence theorem (Eq. \ref{eq:divergence_theorem}) to obtain:

\begin{equation}
  \int_V \frac{\partial \rho}{\partial t} dV = \int_V \nabla \cdot (\rho \mathbf{u}) dV
\end{equation}
Let $\Delta V \to 0$ to integrate and drop $\Delta V$ on both sides to obtain
Eq. \ref{eq:continuity_eulerian}, which is the Eulerian form of the continuity
equation.

\subsubsection{Lagrangian derivation}
\index{Continuity!Lagrangian derivation}

In the Lagrangian frame, we follow a fluid parcel as it moves through space.
Its mass $\rho \Delta V$ is constant by definition, but its density or volume
may change.
Since the mass of the parcel is constant, its Lagrangian derivative is zero:

\begin{equation}
  \frac{d}{dt} (\rho \Delta V) = 0
\end{equation}
Since the mass doesn't change, any change in the density of the parcel must be
balanced by a change in its volume:

\begin{equation}
  \Delta V \frac{d\rho}{dt} + \rho \frac{d\Delta V}{dt} = 0
\end{equation}
Recall that we've already derived the Lagrangian derivative of a volume of the
fluid parcel (Eq. \ref{eq:lagrangian_volume_derivative}), which is the second
term here.
The equation becomes:

\begin{equation}
  \Delta V \frac{d\rho}{dt} + \Delta V \rho \nabla \cdot \mathbf{u} = 0
\end{equation}
Finally, drop $\Delta V$ on both sides to obtain the Lagrangian form of the
continuity equation:

\begin{equation}
  \frac{d\rho}{dt} + \rho \nabla \cdot \mathbf{u} = 0
  \label{eq:continuity_lagrangian}
\end{equation}

Equations \ref{eq:continuity_eulerian} and \ref{eq:continuity_lagrangian} are
two fundamental expressions of the conservation of mass for a fluid.
In one form or another, this equation is a critical component of all weather,
ocean, and climate prediction models.

\subsubsection{Continuity of an incompressible fluid}

Liquids are nearly incompressible, and for them $\frac{d\rho}{dt} = 0$ is a good
approximation.
For an incompressible fluid, the continuity equation simplifies to:

\begin{equation}
  \nabla \cdot \mathbf{u} = 0
  \label{eq:continuity_incompressible}
\end{equation}
Although as simple as it gets, Eq. \ref{eq:continuity_incompressible} is
extremely important in fluid dynamics.

\subsection{Conservation of momentum}

Like the conservation of mass, the conservation of momentum is a fundamental
concept in fluid mechanics.
It allows us to predict how the fluid should accelerate due to its state
(i.e. velocity and density) and due to the forces acting on it.
Together, the continuity and momentum conservation
equations form the core of most fluid prediction models, such as weather, ocean,
and climate prediction models.
We will derive the momentum equation in the remainder of this section.
We'll start from the most basic form first and then incrementally introduce
some common forces, such as the pressure gradient force, gravity, and viscosity.

\subsubsection{The first step}

To derive the momentum conservation equation, we will start from the second
Newton's law, which states that the time rate of change of the momentum of a
fluid particle is equal to the net force acting on it.
For a fluid parcel of volume $\Delta V = \int_V dV$ whose momentum per unit mass
is $\rho \mathbf{u}$, the momentum conservation equation is:

\begin{equation}
  \frac{d}{dt} \int_V \rho \mathbf{u} dV = \int_V \mathbf{F} dV
\end{equation}
where $\mathbf{F}$ is the net force per unit volume acting on the fluid parcel.
Let again the volume parcel be very small such that its density and net force
acting on it are uniform. We have:

\begin{equation}
  \rho \frac{d\mathbf{u}}{dt} \Delta V = \mathbf{F} \Delta V
\end{equation}

\begin{equation}
  \rho \frac{d\mathbf{u}}{dt} = \mathbf{F}
\end{equation}
Recall the Lagrangian derivative operator from Eq. \ref{eq:lagrangian_derivative}
to obtain:

\begin{equation}
  \frac{\partial \mathbf{u}}{\partial t} + (\mathbf{u} \cdot \nabla) \mathbf{u} = \frac{\mathbf{F}}{\rho}
  \label{eq:momentum_eulerian}
\end{equation}
This equation states that the acceleration of a fluid parcel at any fixed point
in space is equal to the net force per unit mass acting on it, divided by the
fluid density.
The second term on the left-hand side is the \textit{advection term}\index{Advection}.
It represents the local acceleration of the fluid parcel due to the properties
of the fluid flow itself.
Consider for example a 1-dimensional flow such that the advection term reduces
to $u \frac{\partial u}{\partial x}$.
Notice that the advection term is zero only in two special cases:
when the velocity is zero or when the velocity is spatially uniform.
In all other cases the advection term is non-zero and contributes to the local
acceleration.

Because the advection term is velocity multiplied by its gradient, it is
\textit{nonlinear}\index{Nonlinear}.
This single property of this term makes accurate analysis and prediction of
fluid flows difficult.
For example, the nonlinear advection term is responsible for the existence of
\textit{chaos}\index{Chaos} in fluid flows, where small differences in initial
conditions lead to vastly different outcomes (in popular culture known as the
\textit{butterfly effect}).
One consequence of this in our daily lives is that weather predictability
is limited to a finite lead time horizon, for example one to two weeks depending
on the weather patterns of interest.
If, however, we could assume that either the velocity or its gradient are so
small that they could be neglected, the equation simplifies significantly
and often allows for analytical solutions.

Back to our equation now.
For a 3-dimensional Cartesian flow where the velocity field is $\mathbf{u} = (u, v, w)$
and net forces are $\mathbf{F} = (F_x, F_y, F_z)$, Eq. \ref{eq:momentum_eulerian}
becomes a system of three equations, one for each component of the velocity field.
Recall from the Lagrangian derivative operator that $(\mathbf{u} \cdot \nabla) \mathbf{u}$
is an operator acting on $\mathbf{u}$ (as opposed to divergence of a gradient).
The $(\mathbf{u} \cdot \nabla)$ operator then expands to
$u\frac{\partial}{\partial x} + v\frac{\partial}{\partial y} + w\frac{\partial}{\partial z}$.
Our vector equations becomes a system of three scalar equations:

\begin{equation}
  \frac{\partial u}{\partial t} + u \frac{\partial u}{\partial x} + v \frac{\partial u}{\partial y} + w \frac{\partial u}{\partial z} = \frac{F_x}{\rho}
\end{equation}

\begin{equation}
  \frac{\partial v}{\partial t} + u \frac{\partial v}{\partial x} + v \frac{\partial v}{\partial y} + w \frac{\partial v}{\partial z} = \frac{F_y}{\rho}
\end{equation}

\begin{equation}
  \frac{\partial w}{\partial t} + u \frac{\partial w}{\partial x} + v \frac{\partial w}{\partial y} + w \frac{\partial w}{\partial z} = \frac{F_z}{\rho}
\end{equation}
Each of the prognostic equations for the velocity components thus has exactly
three advective components that correspond to the gradients of the velocity
in each respective direction.

\subsubsection{Incorporating the forces}

Now we should consider what forces may be acting on the fluid.
We distinguish between two types of forces: surface forces and body forces.
Surface forces act on the surface of the fluid parcel due to the motion of the
fluid molecules, in all directions at that surface.
For example, organized motion of molecules into the surface may cause pressure
on that surface, and the sheared motion of molecules (e.g. if flow is
antiparallel to the surface) may cause shear stress on the surface, leading to
the deformation of the fluid parcel.
In contrast, body forces act remotely (meaning, from a distance) on the entire
volume of the fluid parcel because that parcel is immersed in one or more force fields.
Gravity is one such body force, and it's the only one we'll consider here.
Although in Eq. \ref{eq:momentum_eulerian} we wrote the net force as $\mathbf{F}$,
it's useful to write it as the sum of body forces $\mathbf{F}_b$ and surface
forces $\mathbf{F}_s$:

\begin{equation}
  \frac{\partial \mathbf{u}}{\partial t} + (\mathbf{u} \cdot \nabla) \mathbf{u} = \frac{1}{\rho} (\mathbf{F}_s + \mathbf{F}_b)
\end{equation}

Let's derive the surface forces first.
We want to find out the local change of momentum only due to the surface forces.
Analogous to how the flow through the volume determined the rate of change of
density inside that volume, as we saw in the continuity equation
(Eq. \ref{eq:continuity_eulerian}), the change in momentum inside the volume
is determined by the surface forces acting on the volume (Fig. \ref{fig:momentum1}).

\begin{figure}[h]
  \centering
  \includegraphics[width=0.8\textwidth]{assets/fig_momentum1.png}
  \caption{
    Normal components of the stress tensor $\mathbf{\sigma}$ acting on a fluid parcel.
    Reproduced from \url{https://en.wikipedia.org/wiki/Cauchy_momentum_equation}
    under the CC BY-SA 4.0 license.
  }
  \label{fig:momentum1}
\end{figure}

Mathematically, we can express this change as:

\begin{equation}
  \int_V \mathbf{F_s} dV = \int_S \boldsymbol{\sigma} \cdot d\mathbf{S}
\end{equation}
where $\boldsymbol{\sigma}$ is the second-order stress tensor acting on the
surface $S$ of the fluid parcel.
As before, recall the divergence theorem (Eq. \ref{eq:divergence_theorem}) to obtain:

\begin{equation}
  \int_V \mathbf{F_s} dV = \int_V \nabla \cdot \boldsymbol{\sigma} dV
\end{equation}

\begin{equation}
  \mathbf{F_s} = \nabla \cdot \boldsymbol{\sigma}
\end{equation}
The surface force thus equals the divergence of the stress tensor.
Insert this into Eq. \ref{eq:momentum_eulerian} to get our new form of
the momentum equation:

\begin{equation}
  \frac{\partial \mathbf{u}}{\partial t} + (\mathbf{u} \cdot \nabla) \mathbf{u} =
  \frac{1}{\rho} \nabla \cdot \boldsymbol{\sigma} + \frac{\mathbf{F}_b}{\rho}
\end{equation}
This form of the momentum equation is often called the
\textit{Cauchy momentum equation}\index{Momentum equation!Cauchy}.

Let's now look at what this stress tensor divergence term
$\nabla \cdot \boldsymbol{\sigma}$ is.

\subsubsection{Pressure gradient}

There is a fundamental difference in the meaning of the diagonal and off-diagonal
components of the stress tensor.
The diagonal components of the stress tensor, $\sigma_{xx}$, $\sigma_{yy}$, and
$\sigma_{zz}$, represent the normal stress components, i.e. the force per unit
area acting on a surface element that is oriented in the $x$, $y$, and $z$
directions, respectively.
The off-diagonal components of the stress tensor represent the shear stress
components, each acting on all three surfaces.
For example, $\sigma_{xy}$ represents the shear stress component acting on the
$x$-oriented surface due to the motion of the fluid in the $y$ direction, and
vice versa.
Let's write out the stress tensor in Cartesian coordinates:

\begin{equation}
  \boldsymbol{\sigma} = \begin{bmatrix}
    \sigma_{xx} & \sigma_{xy} & \sigma_{xz} \\
    \sigma_{yx} & \sigma_{yy} & \sigma_{yz} \\
    \sigma_{zx} & \sigma_{zy} & \sigma_{zz}
  \end{bmatrix}
\end{equation}

This tensor can be decomposed into its normal and shear components:

\begin{equation}
  \boldsymbol{\sigma} = -p \mathbf{I} + \boldsymbol{\tau}
  \label{eq:stress_tensor_decomposition}
\end{equation}
where $p$ is the pressure, $\mathbf{I}$ is the identity tensor\index{Tensor!Identity},
and $\boldsymbol{\tau}$ is the deviatoric stress tensor, or, the viscous shear
stress tensor\index{Stress!Shear}.
Written out explicitly in Cartesian coordinates and using Eq.
\ref{eq:stress_tensor_decomposition}, the stress tensor is:

\begin{equation}
  \boldsymbol{\sigma} = \begin{bmatrix}
    -p + \tau_{xx} & \tau_{xy} & \tau_{xz} \\
    \tau_{yx} & -p + \tau_{yy} & \tau_{yz} \\
    \tau_{zx} & \tau_{zy} & -p + \tau_{zz}
  \end{bmatrix}
\end{equation}

The divergence of the stress tensor is then:

\begin{equation}
  \nabla \cdot \boldsymbol{\sigma} = - \nabla p + \nabla \cdot \boldsymbol{\tau}
\end{equation}
Let's insert this into Eq. \ref{eq:momentum_eulerian} to get our new form of
the momentum equation:

\begin{equation}
  \frac{\partial \mathbf{u}}{\partial t} + (\mathbf{u} \cdot \nabla) \mathbf{u} =
  - \frac{1}{\rho} \nabla p + \frac{1}{\rho} \nabla \cdot \boldsymbol{\tau} + \frac{\mathbf{F}_b}{\rho}
  \label{eq:momentum_cauchy_with_shear}
\end{equation}

Pressure is one of the fluid properties that determine its state.
Collective, organized, motion of molecules at a macroscopic scale induces
pressure on a surface and an associated force acting normal to that surface.
Recall that the surface vector is normal to the surface and pointing outward,
and the force acting on the fluid surface is oriented inward, thus the minus sign.

In an ideal, \textit{inviscid} fluid, that is, a fluid that exhibits no viscous
forces, the stress tensor $\boldsymbol{\sigma}$ is only composed of the diagonal
terms (pressure), and the divergence of the stress tensor is zero.
Dropping $\nabla \cdot \boldsymbol{\tau}$ and the body forces $\mathbf{F}_b$ for
now, the Cauchy momentum equation simplifies to:

\begin{equation}
  \frac{\partial \mathbf{u}}{\partial t} + (\mathbf{u} \cdot \nabla) \mathbf{u} =
  - \frac{1}{\rho} \nabla p
  \label{eq:momentum_euler}
\end{equation}
This form of the momentum equation is often called the \textit{Euler equation}.
\index{Momentum equation!Euler}

\subsubsection{Viscous forces}

Now, let's look at the shear stress tensor divergence $\nabla \cdot \boldsymbol{\tau}$.
Written out explicitly as a matrix of all its components, $\boldsymbol{\tau}$ is:

\begin{equation}
  \boldsymbol{\tau} = \begin{bmatrix}
    \tau_{xx} & \tau_{xy} & \tau_{xz} \\
    \tau_{yx} & \tau_{yy} & \tau_{yz} \\
    \tau_{zx} & \tau_{zy} & \tau_{zz}
  \end{bmatrix}
\end{equation}
The diagonal components of the deviatoric stress tensor are the normal stresses,
while the off-diagonal components are the shear stresses.
The normal stresses are non-zero only in compressible fluids ($\nabla \cdot \mathbf{u} \neq 0$),
while the shear stresses are zero in non-viscous flows.
The divergence of this tensor, written out explicitly as a matrix of all its
components, is:

\begin{equation}
  \nabla \cdot \boldsymbol{\tau} = \begin{bmatrix}
    \frac{\partial \tau_{xx}}{\partial x} + \frac{\partial \tau_{yx}}{\partial y} + \frac{\partial \tau_{zx}}{\partial z} \\
    \frac{\partial \tau_{xy}}{\partial x} + \frac{\partial \tau_{yy}}{\partial y} + \frac{\partial \tau_{zy}}{\partial z} \\
    \frac{\partial \tau_{xz}}{\partial x} + \frac{\partial \tau_{yz}}{\partial y} + \frac{\partial \tau_{zz}}{\partial z}
  \end{bmatrix}
\end{equation}

Now, write out \ref{eq:momentum_cauchy_with_shear} as a system of three scalar
equations, one for each component of the velocity field, and insert the shear
stress divergence terms to get:

\begin{equation}
  \frac{\partial u}{\partial t} + 
  u \frac{\partial u}{\partial x} + 
  v \frac{\partial u}{\partial y} + 
  w \frac{\partial u}{\partial z} = 
  - \frac{1}{\rho} \frac{\partial p}{\partial x} + 
  \frac{1}{\rho} \left( \frac{\partial \tau_{xx}}{\partial x} + \frac{\partial \tau_{yx}}{\partial y} + \frac{\partial \tau_{zx}}{\partial z} \right) + 
  \frac{F_x}{\rho}
\end{equation}

\begin{equation}
  \frac{\partial v}{\partial t} + 
  u \frac{\partial v}{\partial x} + 
  v \frac{\partial v}{\partial y} + 
  w \frac{\partial v}{\partial z} = 
  - \frac{1}{\rho} \frac{\partial p}{\partial y} + 
  \frac{1}{\rho} \left( \frac{\partial \tau_{xy}}{\partial x} + \frac{\partial \tau_{yy}}{\partial y} + \frac{\partial \tau_{zy}}{\partial z} \right) + 
  \frac{F_y}{\rho}
\end{equation}

\begin{equation}
  \frac{\partial w}{\partial t} + 
  u \frac{\partial w}{\partial x} + 
  v \frac{\partial w}{\partial y} + 
  w \frac{\partial w}{\partial z} = 
  - \frac{1}{\rho} \frac{\partial p}{\partial z} + 
  \frac{1}{\rho} \left( \frac{\partial \tau_{xz}}{\partial x} + \frac{\partial \tau_{yz}}{\partial y} + \frac{\partial \tau_{zz}}{\partial z} \right) + 
  \frac{F_z}{\rho}
\end{equation}
Each of the prognostic equations for the velocity components thus has exactly
one pressure gradient and two shear stress gradient terms, all arising from the
surface forces.

Experimentally, it was found that the viscous shear stress tensor $\boldsymbol{\tau}$
is proportional to the gradient of the velocity field, i.e. $\boldsymbol{\tau} = \mu \nabla \mathbf{u}$.
This property of the fluid makes it a so-called \textit{Newtonian fluid}\index{Fluid!Newtonian}.
The proportionality constant $\mu$ is the dynamic viscosity and depends on the
fluid properties and temperature.
Inserting this into Eq. \ref{eq:momentum_cauchy_with_shear} and assuming
incompressibility, we get:

\begin{equation}
  \frac{\partial \mathbf{u}}{\partial t} + (\mathbf{u} \cdot \nabla) \mathbf{u} =
  - \frac{1}{\rho} \nabla p + \frac{1}{\rho} \nabla \cdot (\mu \nabla \mathbf{u}) +
  \frac{\mathbf{F}_b}{\rho}
\end{equation}
We can further simplify this equation by assuming that the viscosity is constant
and that the flow is incompressible.
This allows us to neglect the viscous stress gradient term, leading to the
\textit{Navier-Stokes equation}\index{Momentum equation!Navier-Stokes}.

\begin{equation}
  \frac{\partial \mathbf{u}}{\partial t} + (\mathbf{u} \cdot \nabla) \mathbf{u} =
  - \frac{1}{\rho} \nabla p + \nu \nabla^2 \mathbf{u} +
  \frac{\mathbf{F}_b}{\rho}
  \label{eq:momentum_navier_stokes}
\end{equation}
where $\nu = \frac{\mu}{\rho}$ is the kinematic viscosity.
The $\nabla^2$ operator is often called the \textit{Laplacian}\index{Laplacian}.
It is a second-order differential operator that appears in many partial
differential equations, including the heat equation, the wave equation, and
the Laplace equation.
More on these later.

Let's now look at the body forces to conclude our derivation.

\subsection{Gravity}

As we mentioned earlier, gravity is the only body force we'll consider here.
The force of gravity per unit mass is given by $\mathbf{g} = (0, 0, -g)$,
where $g$ is the gravitational acceleration.
Here we're assuming that the gravitational acceleration is constant and always
points downward.
Insert this into Eq. \ref{eq:momentum_navier_stokes} and assuming
incompressibility, we get:

\begin{equation}
  \frac{\partial \mathbf{u}}{\partial t} + (\mathbf{u} \cdot \nabla) \mathbf{u} =
  - \frac{1}{\rho} \nabla p + \mathbf{g} + \nu \nabla^2 \mathbf{u}
\end{equation}

Written out explicitly, we get:

\begin{equation}
  \frac{\partial u}{\partial t} + 
  u \frac{\partial u}{\partial x} + 
  v \frac{\partial u}{\partial y} + 
  w \frac{\partial u}{\partial z} = 
  - \frac{1}{\rho} \frac{\partial p}{\partial x} + \nu \left( \frac{\partial^2 u}{\partial x^2} + \frac{\partial^2 u}{\partial y^2} + \frac{\partial^2 u}{\partial z^2} \right)
\end{equation}

\begin{equation}
  \frac{\partial v}{\partial t} + 
  u \frac{\partial v}{\partial x} + 
  v \frac{\partial v}{\partial y} + 
  w \frac{\partial v}{\partial z} = 
  - \frac{1}{\rho} \frac{\partial p}{\partial y} + \nu \left( \frac{\partial^2 v}{\partial x^2} + \frac{\partial^2 v}{\partial y^2} + \frac{\partial^2 v}{\partial z^2} \right)
\end{equation}

\begin{equation}
  \frac{\partial w}{\partial t} + 
  u \frac{\partial w}{\partial x} + 
  v \frac{\partial w}{\partial y} + 
  w \frac{\partial w}{\partial z} = 
  - \frac{1}{\rho} \frac{\partial p}{\partial z} - g + \nu \left( \frac{\partial^2 w}{\partial x^2} + \frac{\partial^2 w}{\partial y^2} + \frac{\partial^2 w}{\partial z^2} \right)
\end{equation}

\subsection{Exercises}

\begin{enumerate}
  \item Derive the Lagrangian form of the continuity equation from
  the Eulerian form and vice versa. What is the key equation that relates the
  two forms?

  \item Write out the Cauchy, Euler, and Navier-Stokes equations in vector form
  and discuss their similarities and differences.
  Give examples of flows that are well described by each of these equations.

  \item Write a computer program that computes the divergence of a second-order
  tensor in a Cartesian, 3-dimensional coordinate system.
\end{enumerate}

\subsection*{Further reading}

\begin{itemize}
  \item Sections 1.2-1.3 of \textit{EAOD} by Vallis
  \item Sections 4.1-4.6 of \textit{Fluid Mechanics} by Kundu, Cohen, and Dowling
\end{itemize}


%\subsection{Conservation of energy}

%\newpage
%\section{Rotating and stratified flows}

%\newpage
%\section{Shallow water equations}

%\newpage
%\section{Boundary layers}

%\newpage
%\section{Turbulence}

%\newpage
%\section{Surface gravity waves}

\newpage
\appendix

\section{Quick reference}

This section serves a quick reference for the key equations used in this book.\\

\textbf{Gradient:}

\begin{equation}
  \nabla = \frac{\partial}{\partial x} \mathbf{i} + \frac{\partial}{\partial y} \mathbf{j} + \frac{\partial}{\partial z} \mathbf{k}
\end{equation}

\textbf{Divergence:}

\begin{equation}
  \nabla \cdot \mathbf{u} = \frac{\partial u}{\partial x} + \frac{\partial v}{\partial y} + \frac{\partial w}{\partial z}
\end{equation}

\textbf{Curl:}

\begin{equation}
  \nabla \times \mathbf{u} = \left( \frac{\partial w}{\partial y} - \frac{\partial v}{\partial z} \right) \mathbf{i} + \left( \frac{\partial u}{\partial z} - \frac{\partial w}{\partial x} \right) \mathbf{j} + \left( \frac{\partial v}{\partial x} - \frac{\partial u}{\partial y} \right) \mathbf{k}
\end{equation}

\textbf{Laplacian:}

\begin{equation}
  \nabla^2 = \frac{\partial^2}{\partial x^2} + \frac{\partial^2}{\partial y^2} + \frac{\partial^2}{\partial z^2}
\end{equation}

\textbf{Lagrangian derivative operator:}

\begin{equation}
  \frac{d}{dt} = \frac{\partial}{\partial t} + (\mathbf{u} \cdot \nabla)
\end{equation}

\textbf{Continuity, Eulerian form:}

\begin{equation}
  \frac{\partial \rho}{\partial t} + \nabla (\rho \mathbf{u}) = 0
\end{equation}

\textbf{Continuity, Lagrangian form:}

\begin{equation}
  \frac{d\rho}{dt} + \rho \nabla \cdot \mathbf{u} = 0
\end{equation}

\textbf{Momentum, Cauchy:}

\begin{equation}
  \frac{\partial \mathbf{u}}{\partial t} + (\mathbf{u} \cdot \nabla) \mathbf{u} =
  \frac{1}{\rho} \nabla \cdot \boldsymbol{\sigma} + \frac{\mathbf{F}_b}{\rho}
\end{equation}

\textbf{Stress tensor as a combination of pressure and deviatoric stress:}

\begin{equation}
  \boldsymbol{\sigma} = -p \mathbf{I} + \boldsymbol{\tau}
\end{equation}

\textbf{Momentum, Euler:}

\begin{equation}
  \frac{\partial \mathbf{u}}{\partial t} + (\mathbf{u} \cdot \nabla) \mathbf{u} =
  - \frac{1}{\rho} \nabla p
\end{equation}

\textbf{Momentum, Navier-Stokes:}

\begin{equation}
  \frac{\partial \mathbf{u}}{\partial t} + (\mathbf{u} \cdot \nabla) \mathbf{u} =
  - \frac{1}{\rho} \nabla p + \nu \nabla^2 \mathbf{u} + \frac{\mathbf{F}_b}{\rho}
\end{equation}

\textbf{Momentum, with body force (gravity):}

\begin{equation}
  \frac{\partial \mathbf{u}}{\partial t} + (\mathbf{u} \cdot \nabla) \mathbf{u} =
  - \frac{1}{\rho} \nabla p + \mathbf{g} + \nu \nabla^2 \mathbf{u}
\end{equation}

\textbf{Momentum, Navier-Stokes, in scalar form:}

\begin{equation}
  \frac{\partial u}{\partial t} + 
  u \frac{\partial u}{\partial x} + 
  v \frac{\partial u}{\partial y} + 
  w \frac{\partial u}{\partial z} = 
  - \frac{1}{\rho} \frac{\partial p}{\partial x} + \nu \left( \frac{\partial^2 u}{\partial x^2} + \frac{\partial^2 u}{\partial y^2} + \frac{\partial^2 u}{\partial z^2} \right)
\end{equation}

\begin{equation}
  \frac{\partial v}{\partial t} + 
  u \frac{\partial v}{\partial x} + 
  v \frac{\partial v}{\partial y} + 
  w \frac{\partial v}{\partial z} = 
  - \frac{1}{\rho} \frac{\partial p}{\partial y} + \nu \left( \frac{\partial^2 v}{\partial x^2} + \frac{\partial^2 v}{\partial y^2} + \frac{\partial^2 v}{\partial z^2} \right)
\end{equation}

\begin{equation}
  \frac{\partial w}{\partial t} + 
  u \frac{\partial w}{\partial x} + 
  v \frac{\partial w}{\partial y} + 
  w \frac{\partial w}{\partial z} = 
  - \frac{1}{\rho} \frac{\partial p}{\partial z} - g + \nu \left( \frac{\partial^2 w}{\partial x^2} + \frac{\partial^2 w}{\partial y^2} + \frac{\partial^2 w}{\partial z^2} \right)
\end{equation}

\printindex

\end{document}
