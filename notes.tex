\documentclass[12pt]{article}

\usepackage[margin=1in,footskip=0.25in]{geometry}
\usepackage{amsmath}
\usepackage{amssymb}
\usepackage{graphicx}

\begin{document}

\title{Fluid Mechanics for Ocean Scientists}
\author{Milan Curcic}
\date{}

\maketitle

\tableofcontents

\newpage
\section{Introduction}

\subsection{What will you learn in this course}

\subsection{Grading}

\subsection{Reference textbooks}

\section{Review of vector calculus}

In this section we will review the necessary concepts from vector calculus that
we will use in this course.
These include:
scalars, vectors and tensors;
gradient, divergence, and curl;
line, surface, and volume integrals;
and the Gauss and Stokes theorems.

\subsection{Scalars, vectors, and tensors}

In this course we will mainly use three types of quantities to describe fluid
properties: \textbf{scalars}, \textbf{vectors}, and \textbf{tensors}.

\textbf{Scalars} are completely described by their magnitude.
Examples of scalars are temperature, pressure, or density.
A value of 290 K, for example, completely describes the temperature of a fluid
at some point in space and time.
We will write them in equations using italics, e.g. $T$, $p$, $\rho$.

\textbf{Vectors} have both magnitude and direction.
Examples of vector are velocity, acceleration, or force.
In 3-dimensional Cartesian space with coordinates $(x, y, z)$, for example,
vector $\mathbf{u}(x,y,z)$ can be described by its components

\begin{equation}
  \mathbf{u} = (u_x, u_y, u_z)
\end{equation}
where $u_x$, $u_y$, and $u_z$ (each a scalar) are the components of $\mathbf{u}$
in the $x$, $y$, and $z$ directions, respectively.
We will write vectors in equations using boldface, e.g. $\mathbf{u}$,
$\mathbf{a}$, or $\mathbf{F}$.

\textbf{Tensors} have magnitude, direction, and orientation, such as stress or
strain.
They are vectors that act on each respective surface orthogonal to the direction
of the tensor.
In 3-dimensional space, for example, a stress tensor can be described as:

\begin{equation}
    \mathbf{\tau} =
  \begin{bmatrix}
    \tau_{xx} & \tau_{xy} & \tau_{xz} \\
    \tau_{yx} & \tau_{yy} & \tau_{yz} \\
    \tau_{zx} & \tau_{zy} & \tau_{zz}
  \end{bmatrix}
\end{equation}

It may be useful to think of scalars as 0th-order tensors, vectors as 1st-order
tensors, and tensors as 2nd-order tensors.

\subsection{Unit vectors}

Unit vectors are vectors with magnitude of 1.
A popular notation for unit vectors in Cartesian coordinates is $\mathbf{i}$,
$\mathbf{j}$, and $\mathbf{k}$, which point in the $x$, $y$, and $z$ directions,
respectively.
So, a vector $\mathbf{u}$ can be written as

\begin{equation}
  \mathbf{u} = u_x \mathbf{i} + u_y \mathbf{j} + u_z \mathbf{k}
\end{equation}

\subsection{Vector operations}

\subsubsection{Dot product}

The dot product of two vectors $\mathbf{a}$ and $\mathbf{b}$ is an element-wise
sum of their components:

\begin{equation}
  \mathbf{a} \cdot \mathbf{b} = a_1 b_1 + a_2 b_2 + a_3 b_3
\end{equation}

\subsubsection{Cross product}

The cross product of two vectors $\mathbf{a}$ and $\mathbf{b}$ is defined as

\begin{equation}
  \mathbf{a} \times \mathbf{b} = (a_y b_z - a_z b_y) \mathbf{i} +
    (a_z b_x - a_x b_z) \mathbf{j} + (a_x b_y - a_y b_x) \mathbf{k}
\end{equation}

\subsection{Gradient, divergence, and curl}

Now, we will introduce another new operator that builds on top of previous
concepts to describe how scalar and vector fields change in space.
This operator is called \textbf{nabla} and is denoted by $\nabla$:

\begin{equation}
  \nabla = \frac{\partial}{\partial x} \mathbf{i} +
    \frac{\partial}{\partial y} \mathbf{j} +
    \frac{\partial}{\partial z} \mathbf{k}
\end{equation}

\subsubsection{Gradient}

The gradient of a scalar field $T$ is a vector field that points in the
direction of the greatest rate of increase of $T$.
It is denoted by $\nabla T$ and is defined as

\begin{equation}
  \nabla T = \frac{\partial T}{\partial x} \mathbf{i} +
    \frac{\partial T}{\partial y} \mathbf{j} +
    \frac{\partial T}{\partial z} \mathbf{k}
\end{equation}

\subsubsection{Divergence}

The divergence of a vector field $\mathbf{u}$ is a scalar field that describes
the rate at which the vector field flows out of a point.
It is denoted by $\nabla \cdot \mathbf{u}$ and is defined as

\begin{equation}
  \nabla \cdot \mathbf{u} = \frac{\partial u_x}{\partial x} +
    \frac{\partial u_y}{\partial y} + \frac{\partial u_z}{\partial z}
\end{equation}

\subsubsection{Curl}

The curl of a vector field $\mathbf{u}$ is a vector field that describes the
rotation of the vector field.
It is denoted by $\nabla \times \mathbf{u}$ and is defined as

\begin{equation}
  \nabla \times \mathbf{u} = \left( \frac{\partial u_z}{\partial y} -
    \frac{\partial u_y}{\partial z} \right) \mathbf{i} +
    \left( \frac{\partial u_x}{\partial z} -
    \frac{\partial u_z}{\partial x} \right) \mathbf{j} +
    \left( \frac{\partial u_y}{\partial x} -
    \frac{\partial u_x}{\partial y} \right) \mathbf{k}
\end{equation}

\end{document}